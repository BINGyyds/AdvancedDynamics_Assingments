\documentclass[a4paper,12pt]{article}
\usepackage[UTF8]{ctex} % 支持中文
\usepackage{amsmath}
\usepackage{amssymb}
\usepackage{graphicx}
\usepackage{geometry}
\usepackage{float}
\usepackage{xcolor}
\usepackage{tikz} % 用于绘制占位符

% 页面边距设置
\geometry{left=2.5cm, right=2.5cm, top=2.5cm, bottom=2.5cm}

% 定义图片占位符命令
\newcommand{\imageplaceholder}[2]{
    \begin{figure}[H]
        \centering
        \begin{tikzpicture}
            \draw[thick, dashed] (0,0) rectangle (10,5);
            \node at (5,2.5) {此处为插图: #1};
        \end{tikzpicture}
        \caption*{#2} % 使用无编号标题,或者根据需要使用 \caption
    \end{figure}
}

\begin{document}

% 封面/页头信息
\noindent {\Large \textbf{高等动力学——作业}}

\vspace{0.5cm}

\begin{itemize}
    \item 姓名:李融冰
    \item 学号:2120250024
\end{itemize}

\vspace{0.5cm}

% ----------------------------------------------------------------------
% P 6.36
% ----------------------------------------------------------------------
\section*{P 6.36}

\imageplaceholder{机构示意图}{原图 P6.36 机构示意图}

\subsection*{(1)画自由体图(FBD)}
取分离刚体,其受力有:杆的轴向力 $T$;重力 $mg$;弹簧弹力 $F_k$;阻尼器的阻力 $F_c$。受力图如下:

\imageplaceholder{自由体图}{受力图}

\subsection*{(2)推导动力学方程}
系统共两个自由度,取 $\theta$ 和 $\phi$ 作为广义坐标,首先,可以推导质心距离 OD 面的距离为
\begin{equation}
    h = l \cos\theta + d \cos\phi
\end{equation}
然后,考虑质心速度为
\begin{equation}
    \vec{v}_C = -(l\dot{\theta}\sin\theta + d\dot{\phi}\sin\phi)\vec{i}_1 + (l\dot{\theta}\cos\theta + d\dot{\phi}\cos\phi)\vec{i}_2
\end{equation}
刚体角速度大小为
\begin{equation}
    \omega = \dot{\phi}
\end{equation}
接下来,考虑向量 $\vec{BD}$,根据
\begin{equation}
    \vec{OA} = l \cos\theta \vec{i}_1 + l \sin\theta \vec{i}_2
\end{equation}
\begin{equation}
    \vec{AB} = \begin{bmatrix} \vec{i}_1 & \vec{i}_2 \end{bmatrix} 
    \begin{bmatrix} \cos\phi & -\sin\phi \\ \sin\phi & \cos\phi \end{bmatrix}
    \begin{bmatrix} s_1^* \\ s_2^* \end{bmatrix}
    = (s_1^* \cos\phi - s_2^* \sin\phi)\vec{i}_1 + (s_1^* \sin\phi + s_2^* \cos\phi)\vec{i}_2
\end{equation}
\begin{equation}
    \vec{OD} = w \vec{i}_2
\end{equation}
根据向量加法,可知
\begin{equation}
    \begin{aligned}
    \vec{BD} &= -\vec{AB} - \vec{OA} + \vec{OD} \\
             &= (-s_1^* \cos\phi + s_2^* \sin\phi - l \cos\theta)\vec{i}_1 + (-s_1^* \sin\phi - s_2^* \cos\phi - l \sin\theta + w)\vec{i}_2
    \end{aligned}
\end{equation}
因此,$BD$ 长度为
\begin{equation}
    |BD| = \sqrt{(-s_1^* \cos\phi + s_2^* \sin\phi - l \cos\theta)^2 + (-s_1^* \sin\phi - s_2^* \cos\phi - l \sin\theta + w)^2}
\end{equation}
而 B 点处的速度矢量为
\begin{equation}
    \vec{v}_B = (-l\dot{\theta}\sin\theta - s_1^*\dot{\phi}\sin\phi - s_2^*\dot{\phi}\cos\phi)\vec{i}_1 + (l\dot{\theta}\cos\theta + s_1^*\dot{\phi}\cos\phi - s_2^*\dot{\phi}\sin\phi)\vec{i}_2
\end{equation}
沿阻尼器方向的分量为
\begin{equation}
    \begin{aligned}
    v_{Bd} &= \vec{v}_B \cdot \frac{\vec{BD}}{|BD|} \\
    &= \frac{1}{|BD|} \left[ (l\dot{\theta}\sin\theta + s_1^*\dot{\phi}\sin\phi + s_2^*\dot{\phi}\cos\phi)(s_1^*\cos\phi - s_2^*\sin\phi + l\cos\theta) \right. \\
    &\quad \left. + (l\dot{\theta}\cos\theta + s_1^*\dot{\phi}\cos\phi - s_2^*\dot{\phi}\sin\phi)(-s_1^*\sin\phi - s_2^*\cos\phi - l\sin\theta + w) \right] \\
    &= \dots \quad (\text{此处省略部分中间化简步骤,以原图为准}) \\
    \end{aligned}
\end{equation}
综上,刚体块动能为
\begin{equation}
    T = \frac{1}{2}m \left[ l^2\dot{\theta}^2 + d^2\dot{\phi}^2 + 2ld\dot{\theta}\dot{\phi}\cos(\theta-\phi) \right] + \frac{1}{2}J\dot{\phi}^2
\end{equation}
系统的总势能为
\begin{equation}
    \begin{aligned}
    V &= -mg(l\cos\theta + d\cos\phi) + \frac{1}{2}k|BD|^2 \\
      &= -mg(l\cos\theta + d\cos\phi) \\
      &\quad + \frac{1}{2}k \left[ w^2 - 2w(s_1^*\sin\phi + s_2^*\cos\phi + l\sin\theta) + l^2 \right. \\
      &\quad \left. + s_1^{*2} + s_2^{*2} + s_1^*l\cos(\theta-\phi) + s_2^*l\sin(\theta-\phi) \right]
    \end{aligned}
\end{equation}
拉格朗日量为
\begin{equation}
    \begin{aligned}
    L = T - V &= \frac{1}{2}m \left[ l^2\dot{\theta}^2 + d^2\dot{\phi}^2 + 2ld\dot{\theta}\dot{\phi}\cos(\theta-\phi) \right] \\
    &\quad + \frac{1}{2}J\dot{\phi}^2 + mg(l\cos\theta + d\cos\phi) \\
    &\quad - \frac{1}{2}k \left[ w^2 - 2w(s_1^*\sin\phi + s_2^*\cos\phi + l\sin\theta) + l^2 \right. \\
    &\quad \left. + s_1^{*2} + s_2^{*2} + s_1^*l\cos(\theta-\phi) + s_2^*l\sin(\theta-\phi) \right]
    \end{aligned}
\end{equation}
针对广义坐标的阻尼力
\begin{equation}
    Q_\theta = -c v_{Bd} \frac{l\sin\theta - l\cos\theta + w}{|BD|}
\end{equation}
\begin{equation}
    Q_\phi = -c v_{Bd} \frac{s_1^*\sin\phi - s_1^*\cos\phi + s_2^*\cos\phi + s_2^*\sin\phi}{|BD|}
\end{equation}
根据拉格朗日方程,得到系统的动力学方程为
\begin{equation}
    \begin{aligned}
    \frac{\mathrm{d}}{\mathrm{d}t}\left(\frac{\partial L}{\partial \dot{\theta}}\right) - \frac{\partial L}{\partial \theta} 
    &= m \left[ l^2\ddot{\theta} + ld\ddot{\phi}\cos(\theta-\phi) - ld\dot{\phi}(\dot{\theta}-\dot{\phi})\sin(\theta-\phi) \right] \\
    &\quad + ld\dot{\theta}\dot{\phi}\sin(\theta-\phi) + mgl\sin\theta \\
    &\quad + \frac{1}{2} \left[ 2wl\cos\theta - s_1^*l\sin(\theta-\phi) + s_2^*l\cos(\theta-\phi) \right] \\
    &= Q_\theta
    \end{aligned}
\end{equation}
\begin{equation}
    \begin{aligned}
    \frac{\mathrm{d}}{\mathrm{d}t}\left(\frac{\partial L}{\partial \dot{\phi}}\right) - \frac{\partial L}{\partial \phi} 
    &= m \left[ d^2\ddot{\phi} + ld\ddot{\theta}\cos(\theta-\phi) + ld\dot{\theta}\dot{\phi}\sin(\theta-\phi) \right] + J\ddot{\phi} \\
    &\quad - mld\dot{\theta}\dot{\phi}\sin(\theta-\phi) + mgd\sin\phi \\
    &\quad + \frac{1}{2} \left[ -2w(s_1^*\cos\phi - s_2^*\sin\phi) + s_1^*l\sin(\theta-\phi) - s_2^*l\cos(\theta-\phi) \right] \\
    &= Q_\phi
    \end{aligned}
\end{equation}

\subsection*{(3)求杆的载荷}
在每一个运动瞬时,与杆方向重合的空间固定向量为
\begin{equation}
    \vec{r} = \cos\theta \vec{i}_1 + \sin\theta \vec{i}_2
\end{equation}
刚体质心的速度向量如式(2)所示,因此,每一时刻,质心加速度为
\begin{equation}
    \begin{aligned}
    \vec{a}_C &= -(l\ddot{\theta}\sin\theta + l\dot{\theta}^2\cos\theta + d\ddot{\phi}\sin\phi + d\dot{\phi}^2\cos\phi)\vec{i}_1 \\
              &\quad + (l\ddot{\theta}\cos\theta - l\dot{\theta}^2\sin\theta + d\ddot{\phi}\cos\phi - d\dot{\phi}^2\sin\phi)\vec{i}_2
    \end{aligned}
\end{equation}
因此,质心加速度在杆方向上的投影为
\begin{equation}
    a_r = -l\dot{\theta}^2 \cos(\theta-\phi) + d\ddot{\phi}\sin(\theta-\phi) - d\dot{\phi}^2\cos(\theta-\phi)
\end{equation}
重力在杆方向上的分量为
\begin{equation}
    G_r = mg\cos\theta
\end{equation}
式(7)所示的 $BD$ 在杆方向上的投影为
\begin{equation}
    BD_r = -s_1^* \cos(\phi-\theta) + s_2^* \sin(\phi-\theta) - l + w\sin\theta
\end{equation}
因此,弹簧力与阻尼力在杆方向上的投影分别为
\begin{equation}
    F_{kr} = k BD_r
\end{equation}
\begin{equation}
    F_{cr} = -c v_{Bd} \frac{BD_r}{|BD|}
\end{equation}
根据达朗贝尔原理,杆的载荷为(拉正压负)
\begin{equation}
    T_r = m a_r + G_r + F_{kr} + F_{cr}
\end{equation}

% ----------------------------------------------------------------------
% P 6.37
% ----------------------------------------------------------------------
\section*{P 6.37}

\imageplaceholder{曲面上的杆}{原图 P6.37}

\subsection*{(1)考虑法向接触力做功}
法向力的方向始终与接触点位移的方向始终相互垂直,因此,法向接触力不做功。

\subsection*{(2)考虑切向摩擦力做功}
由于杆在曲面上没有滑动,因此,切向摩擦力与接触点无相对滑动位移,摩擦力不做功。

\subsection*{(3)系统的动能变化}
根据上述讨论,法向接触力和切向摩擦力均不做功,因此,杆的动能改变完全由重力做功引起。因此,当杆的质心高度从 $d_1$ 变化为 $d_2$ 时,系统的动能变化为
$$ \Delta K = Mg(d_1 - d_2) $$
由上式可见,系统的动能变化仅仅由系统的质心高度变化决定。

\subsection*{(4)机械能是否守恒}
系统的运动过程中只有重力做功,能量在杆的动能与重力势能之间相互转化,因此机械能守恒。

\subsection*{(5)角动量是否守恒}
在系统运动的某一时刻,杆所受合力对接触点的力矩,等于重力的力矩不等于 0,因此,杆的角动量不守恒。

% ----------------------------------------------------------------------
% P 6.40
% ----------------------------------------------------------------------
\section*{P 6.40}

\imageplaceholder{双杆机构}{Fig. 6.32. Two-bar mechanism.}

\subsection*{(1)推导动力学方程}
系统只有一个自由度,因此,选择广义坐标 $\theta_1$。左侧杆的角速度为
\begin{equation}
    \omega_L = \dot{\theta}_1
\end{equation}
B 点处瞬时速度沿杆分量大小为
\begin{equation}
    \dot{\theta}_1 L_1 \sin(\theta_1 + \theta_2)
\end{equation}
B 点处瞬时速度垂直于杆分量大小为
\begin{equation}
    \dot{\theta}_1 L_1 \cos(\theta_1 + \theta_2)
\end{equation}
在 A 点处,瞬时速度只有沿杆分量,由此可得,右侧杆的角速度为
\begin{equation}
    \omega_R = \dot{\theta}_2 = \frac{\dot{\theta}_1 L_1 \cos(\theta_1 + \theta_2)}{w}
\end{equation}
根据余弦定理得到
\begin{equation}
    w = \sqrt{L_1^2 + d^2 - 2L_1 d \cos\theta_1}
\end{equation}
从而得到导数关系
\begin{equation}
    \frac{\partial w}{\partial \theta_1} = \frac{L_1 d \sin\theta_1}{w}
\end{equation}
根据正弦定理,可得到
\begin{equation}
    \sin\theta_2 = \frac{L_1}{w} \sin\theta_1
\end{equation}
由于 $\theta_1, \theta_2$ 均为三角形内角,因此
\begin{equation}
    \cos\theta_2 = \sqrt{1 - \sin^2\theta_2} = \sqrt{\frac{d^2}{w^2} - \frac{2L_1d}{w^2}\cos\theta_1 + \frac{L_1^2}{w^2}\cos^2\theta_1} = \frac{d}{w} - \frac{L_1}{w}\cos\theta_1
\end{equation}
由此得到导数关系
\begin{equation}
    \dot{\theta}_2 = \frac{L_1 \cos\theta_1}{w \cos\theta_2}\dot{\theta}_1 - \frac{d\sin^2\theta_1}{w \cos\theta_2}\dot{\theta}_1 = \dots = \frac{L_1 \cos(\theta_1 + \theta_2)}{w} \dot{\theta}_1
\end{equation}
上式与式(29)的结果一致。右侧杆质心速度大小为
\begin{equation}
    v_c = \sqrt{\dot{\theta}_1^2 L_1^2 \sin^2(\theta_1 + \theta_2) + \left(w - \frac{L_2}{2}\right)^2 \frac{\dot{\theta}_1^2 L_1^2 \cos^2(\theta_1 + \theta_2)}{w^2}}
\end{equation}
因此系统的总动能为
\begin{equation}
    T = \frac{1}{6}m_1 L_1^2 \dot{\theta}_1^2 + \frac{1}{6}m_2 \frac{L_2^2 L_1^2}{w^2} \cos^2(\theta_1 + \theta_2)\dot{\theta}_1^2 + \frac{1}{2}m_2 L_1^2 \dot{\theta}_1^2 - \frac{1}{2}m_2 \frac{L_2 L_1^2}{w} \cos^2(\theta_1 + \theta_2)\dot{\theta}_1^2
\end{equation}
系统的势能为
\begin{equation}
    V = m_1 g \frac{L_1 \sin\theta_1}{2} + m_2 g \left( w - \frac{L_2}{2} \right)\sin\theta_2 + \frac{1}{2}k(L_2 - w)^2
\end{equation}
系统的拉格朗日量 $L = T - V$,代入拉格朗日方程得到(公式较长,从略,详见原文档 Eq 38)。

\subsection*{(2)无量纲化与数值求解}
将式(38)左右两侧同乘下式
\begin{equation}
    \frac{1}{(m_1+m_2)L_1^2} \cdot \frac{(m_1+m_2)}{k}
\end{equation}
从而得到如下无量纲方程(Eq 40)。

\subsection*{(3)角度 $\theta_1, \theta_2$ 的变化}
\imageplaceholder{角度变化曲线}{角度变化}

\subsection*{(3)无量纲角速度 $\theta_1', \theta_2'$ 的变化}
\imageplaceholder{角速度变化曲线}{角速度变化}

\subsection*{(4)无量纲角加速度 $\theta_1'', \theta_2''$ 的变化}
\imageplaceholder{角加速度变化曲线}{角加速度变化}
角加速度 $\theta_2''$ 按照下式计算为:
\begin{equation}
    \theta_2'' = \frac{\theta_1'' \cos(\theta_1+\theta_2)}{\bar{w}} - \dots \quad (\text{Eq 41})
\end{equation}

\subsection*{(5)弹簧伸长量 $\bar{\Delta}$ 的变化}
\imageplaceholder{弹簧伸长量曲线}{弹簧伸长量变化}

\subsection*{(6)无量纲摩擦力 $\bar{F^f}$ 和反力 $\bar{S}$ 的变化}
\begin{equation}
    \bar{F^f} = 2\xi \theta_1' \sin(\theta_1 + \theta_2)
\end{equation}
\begin{equation}
    \bar{S} = \frac{\mu_2 \bar{L}_2^2 \theta_2'' - \dots}{3\bar{w}} \quad (\text{Eq 43})
\end{equation}
\imageplaceholder{摩擦力和反力曲线}{摩擦力和反力}

\subsection*{(7)耗散能量的变化}
\begin{equation}
    D_k = \frac{2d\sin\left(\frac{\theta_{1,k}^2 - \theta_{1,k-1}^2}{2}\right)}{w_k + w_{k-1}} \xi (\theta_{1,k}' + \theta_{1,k-1}') \sin\dots \quad (\text{Eq 44})
\end{equation}
\imageplaceholder{耗散能量变化曲线}{耗散能量变化}

\subsection*{(8)能量守恒验证}
\imageplaceholder{各类能量变化曲线}{各能量变化}

% ----------------------------------------------------------------------
% P 10.17
% ----------------------------------------------------------------------
\section*{P 10.17}

\imageplaceholder{小车摆杆系统}{原图 P10.17}

\subsection*{(1)推导动力学方程}
系统的总动能为
\begin{equation}
    T = \frac{1}{2}M\dot{x}^2 + \frac{1}{2}m\left( \dot{r}_{A1} - \frac{1}{2}\dot{\theta}L\sin\theta \right)^2 + \frac{1}{2}m\left( \dot{r}_{A2} + \frac{1}{2}\dot{\theta}L\cos\theta \right)^2 + \frac{1}{24}mL^2\dot{\theta}^2
\end{equation}
系统的势能为
\begin{equation}
    V = \frac{1}{2}kx^2 - \frac{1}{2}mgL\cos\theta - mg r_{A1}
\end{equation}
拉格朗日量的变分为
\begin{equation}
    \delta L = \dots \quad (\text{Eq 47})
\end{equation}
阻尼力虚功为
\begin{equation}
    \delta W = -c \dot{x} \delta x
\end{equation}
拉格朗日乘子项为
\begin{equation}
    \delta(C\lambda) = \delta r_{A1} \lambda_1 + (\delta r_{A2} - \delta x)\lambda_2
\end{equation}
根据哈密顿变分原理
\begin{equation}
    \int_{t_0}^{t_f} (\delta L + \delta W + \delta(C\lambda)) dt = 0
\end{equation}
得到动力学方程为(矩阵形式):
\begin{equation}
    \begin{bmatrix}
    M & & & & & 1 \\
    & \frac{1}{3}mL^2 & -\frac{1}{2}L\sin\theta m & \frac{1}{2}L\cos\theta m & & \\
    & -\frac{1}{2}mL\sin\theta & m & & -1 & \\
    & \frac{1}{2}mL\cos\theta & & m & & -1 \\
    & & -1 & & & \\
    1 & & & -1 & & 
    \end{bmatrix}
    \begin{bmatrix}
    \ddot{x} \\ \ddot{\theta} \\ \ddot{r}_{A1} \\ \ddot{r}_{A2} \\ \lambda_1 \\ \lambda_2
    \end{bmatrix}
    = 
    \begin{bmatrix}
    -kx - c\dot{x} \\
    -\frac{1}{2}mgL\sin\theta \\
    \frac{1}{2}m\dot{\theta}^2 L\cos\theta + mg \\
    \frac{1}{2}m\dot{\theta}^2 L\sin\theta \\
    0 \\ 
    0
    \end{bmatrix}
\end{equation}

\subsection*{(2)说明拉格朗日乘子的物理意义}
拉格朗日乘子的物理意义为铰接点 A 处,确保杆与质量块铰接在一起的约束内力,并且 $\lambda_1$ 对应 $\vec{i}_1$ 方向的约束力分量,$\lambda_2$ 对应 $\vec{i}_2$ 方向的约束力分量。

\subsection*{(3)绘制质量块位移 $x$ 的变化图}
\imageplaceholder{位移变化}{小车位移随时间变化}

\subsection*{(4)绘制摆杆角度 $\theta$ 的变化图}
\imageplaceholder{角度变化}{杆角度随时间变化}

\subsection*{(5)绘制杆端点的轨迹图}
\imageplaceholder{端点轨迹}{杆端点轨迹}

\subsection*{(6)绘制质量块速度 $\dot{x}$ 的变化图}
\imageplaceholder{速度变化}{小车速度随时间变化}

\subsection*{(7)绘制杆摆动角速度 $\dot{\theta}$ 的变化图}
\imageplaceholder{角速度变化}{杆角速度随时间变化}

\subsection*{(8)绘制系统的能量变化曲线}
系统的动能和势能可以直接由每一时刻的位置、速度、加速度进行计算,而系统在两个时刻之间的阻尼耗散能量可以由以下公式近似计算
\begin{equation}
    D_k = \frac{\dot{x}_k + \dot{x}_{k-1}}{2}(x_k - x_{k-1})
\end{equation}
\imageplaceholder{能量变化曲线}{能量随时间变化}
从图中可见,系统能量守恒,耗散能量不断增加。

\subsection*{(9)绘制 A 点处内力变化}
\imageplaceholder{内力变化曲线}{内力随时间变化}

\end{document}